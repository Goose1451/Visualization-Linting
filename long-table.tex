
\begin{longtable}{p{3cm}p{14cm}}
  \caption{FILL IN CAPTION}

  \\\hbox{\normalsize{\textbf{CURATING ERRORS}}}&\\ \\
  \normalsize{Error} & \normalsize{Mirage}\\ \hline
   \rowcolor{colora}Forgotten Population or Missing Dataset  & We expect that datasets fully cover or describe phenomena of interest. However, structural, political, and societal biases can result in over- or under-sampling of populations or problems of importance. This mismatch in coverage can hide crucial concerns about the possible scope of our analyses. \cite{missingdatasets, dignazio2019draft}\\
 \rowcolor{colora-opaque}Missing or \newline Repeated Records  & We often assume that we have one and only one entry for each datum. However, errors in data entry or integration can result in missing or repeated values that may result in inaccurate aggregates or groupings (see \figref{fig:misspelling}). \cite{kim2003taxonomy} \\
 \rowcolor{colora}Outliers  & Many forms of analysis assume data have similar magnitudes and were generated by similar processes. Outliers, whether in the form of erroneous or unexpectedly extreme values, can greatly impact aggregation and discredit the assumptions behind many statistical tests and summaries. \cite{kim2003taxonomy} \\
 \rowcolor{colora-opaque}Spelling Mistakes  & Columns of strings are often interpreted as categorical data for the purposes of aggregation. If interpreted in this way, typos or inconsistent spelling and capitalization can create spurious categories, or remove important data from aggregate queries. \cite{wang2019uni}\\
 \rowcolor{colora}Unidentified Functional Dependencies & Visualization might show a strong relationship between variables, when in fact that relationship is part of the data composition (eg A+B = C, plot A vs C) \cite{wang2019uni}\\
 \rowcolor{colora-opaque}Drill-down Bias  & We assume that the order in which we investigate our data should not impact our conclusions. However, by filtering on less explanatory or relevant variables first, the full scope of the impact of later variables can be hidden. This results in insights that address only small parts of the data, when they might be true of the larger whole. \cite{lee2019avoiding}\\

  \\\hbox{\normalsize{\textbf{WRANGLING ERRORS}}}&\\ \\
  \normalsize{Error} & \normalsize{Mirage}\\ \hline
   \rowcolor{colorb}Differing Number \newline of Records by \newline Group  & Certain summary statistics, including aggregates, are sensitive to sample size. However, the number of records aggregated into a single mark can very dramatically. This mismatch can mask this sensitivity and problematize per-mark comparisons; when combined with differing levels of aggregation, it can result in counter-intuitive results such as Simpson's Paradox. \cite{guo2017you}\\
 \rowcolor{colorb-opaque}Cherry Picking & Filtering and subsetting are meant to be tools to remove irrelevant data, or allow the analyst to focus on a particular area of interest. However, if this filtering is too aggressive, or if the analyst focuses on individual examples rather than the general trend, this cherry-picking can promote erroneous conclusions or biased views of the relationships between variables. \cite{few2019loom}\\
 \rowcolor{colorb}Dominating Outlier Wrecks Scale & The magnitude of a real trend might be nullified or made invisible, causing the reader to miss a real data trend.  \cite{kindlmann2014algebraic}\\
 \rowcolor{colorb-opaque}Higher Noise than Effect Size & A high variability or noise in can mask the real information being presented, causing readers to identify non-existent trends. WICKAM LINEUPS?\\
 \rowcolor{colorb}Analyst Degrees of Freedom & Most analytics tools offer a variety of ways of modeling, subsetting, or comparing data, giving the analyst a tremendous amount of freedom. These ``researcher degrees of freedom''~\cite{} can create conclusions that are highly idiosyncratic to the choices made by the analyst, or in a malicious sense promote ``p-hacking'' where the analyst searches through the parameter space in order to find the best support for a pre-ordained conclusion, regardless of the strength of the evidence in the data. A related issue is the ``multiple comparisons problem'' where the analyst makes \emph{so many} choices that at least one, just by happenstance, is likely to appear significant, even if there is no strong signal in the data. \cite{gelman2013garden,pu2018garden,zgraggen2018investigating}\\
 \rowcolor{colorb-opaque}Outliers Combined with Wrong Aggregation Type & Using extremal aggregates (such as min/max) will likely cause misinterpretation of bars as readers tend to assume bars show sums. (I THINK THIS WHAT WITHIN BAR SAYS) \cite{newman2012bar}\\
 \rowcolor{colorb}Aggregates Mask Second Order Statistics  & We often assume that the presentation given to us captures the critical details of the data being considered. Unfortunately highly reductive measures, like averages, can mask second order statistics, which often carry critical information about the variance or distribution, which is masked through simple central tendencies.  \cite{wall2017warning, few2019loom, matejka2017same, anscombe1973graphs, salimi2018bias}\\
 \rowcolor{colorb-opaque}Confusing Imputation  & There are many strategies for dealing with missing or incomplete data, including the imputation of new values. How values are imputed, and then how these imputed values are visualized in the context of the rest of the data, can impact how the data are perceived, in the worst case creating spurious trends or group differences that are merely artifacts of how missing values are handled prior to visualization. \cite{song2018s}\\
 \rowcolor{colorb}Sampling Rate Errors  & Precieved trends in distributions are often subject to the sampling rate at which the underlying data has been curated. This can be problematic as an apparent trend may be an artifact of the sampling rate rather than the data (as is the case visualizations that do not follow the rates suggested by the Nyquist frequency). \cite{kindlmann2014algebraic}\\
 \rowcolor{colorb-opaque}Data Classes Distinctions Ignored & Multiple distinct data classes presented as the same can cause errors in the inference of trends. \cite{anand2015automatic}\\

  \\\hbox{\normalsize{\textbf{VISUALIZING ERRORS}}}&\\ \\
  \normalsize{Error} & \normalsize{Mirage}\\ \hline
   \rowcolor{colorc}Non-sequitur \newline Visualizations  & Readers expect graphics that appear to be charts to be a mapping between data and image. Visualizations being used as decoration (in which the marks are not related to data) present non-information that might be mistaken for real information. \cite{correll2017black}\\
 \rowcolor{colorc-opaque}Misunderstand Area as Quantity  & The use of area encoded marks assumes readers will be able to visually compare those areas. Area encoded marks are often misunderstood as encoding length which can cause ambiguity about interpretation of magnitude. \cite{pandey2015deceptive, correll2017black}\\
 \rowcolor{colorc}Non-discriminable Colors  & The use of color as a data-encoding channel presumes the perceptual discriminability of colors. Poorly chosen color palettes, especially when marks are small or cluttered (cite), can result in ambiguity about which marks belong to which color classes. JND CITATION?\\
 \rowcolor{colorc-opaque}Overplotting  & We expect to be able to clearly identify individual marks, and expect that one visual mark corresponds to a single value or aggregated value. Yet overlapping marks can hide internal structures in the distribution or disguise potential data quality issues, as in \figref{fig:opacity-permute}. \cite{correll2018looks,mayorga2013splatterplots,micallef2017towards}\\
 \rowcolor{colorc}Singularities  & In chart types, such as line series or parallel coordinates plots, many data series can converge into a single point in visual space. Without intervention, viewers can have issues discriminating between which series takes which path after such a singularity. \cite{kindlmann2014algebraic}\\
 \rowcolor{colorc-opaque}Latent Variables Missing & When communicating information about the relationship between two variables, we assume that we have all relevant data. However, in many cases a latent variable has been excluded from the chart, promoting a spurious or non-causative relationship (for instance, both drowning deaths and ice cream sales are tightly correlated, but are related by a latent variable of external temperature). \\
 \rowcolor{colorc}Wrong/Missing Aggregation & AM: I THINK THIS IS A THING BUT WE DON'T REALLY HAVE A DESCRIPTION OF IT \\
 \rowcolor{colorc-opaque}Flipped Axes  & There are conventional expectations for the meaning of axis directions (e.g., time as left to right). When a design ignores those expectations, it can result in a perceived inversion of the trend.  \cite{pandey2015deceptive, correll2017black}\\
 \rowcolor{colorc}Scale Extents Larger than Range of the Data & Differences compressed and visual variability is removed. \cite{cleveland1982variables}\\
 \rowcolor{colorc-opaque}Non-linear Scales & Can cause readers to inaccurately correlate variables and cluster values. \cite{pandey2015deceptive}\\
 \rowcolor{colorc}Truncated/Expanded Axes & Axes constructed in an unintuitive manner may hide variance or cause errors in sorting marks or characterizing the distribution.  The choice of where to place the visual zero line of an axis presumes that XXX. Selections of overly truncated or expanded axes can cause exaggerated effect sizes, which can lead to erroneous judgements (such as in XXX).  \cite{pandey2015deceptive, correll2017black, cleveland1982variables, ritchie2019lie, correll2019truncating}\\
 \rowcolor{colorc-opaque}Base Rate Masquerading as Data  & Visualizations comparing rates are often assumed to show the relative rate, rather than the absolute rate. Yet, many displays give prominence to these absolute or base rates (such as population in choropleth maps) rather than encoded variable, causing the reader to understand this base rate as the data rate.  \cite{correll2016surprise}\\
 \rowcolor{colorc}Inappropriate Semantic Color Scale  & Data-encoded colors can collide with semantically meaningful colors, such as green on a map indicating a forest. This collision can cause readers to mistake the rendered data for associated with that color-connection. CITATION?\\
 \rowcolor{colorc-opaque}Within-the-Bar-Bias & The filled in area underneath a bar chart does not communicate any information about likelihood. However, viewers often erroneously presume that values inside the visual area of the bar are liklier or more probable than values outside of this region. \cite{newman2012bar}\\
 \rowcolor{colorc}Highlighting/Downplaying Outliers & Over highlighting outliers can mask important data features, however so can ignoring them. CITATION?\\
 \rowcolor{colorc-opaque}Clipped Outliers  & Charts are often assumed to show the full extent of their input data. A chosen domain might exclude meaningful outliers, causing some trends in the data to be invisible to the reader. \\
 \rowcolor{colorc}Continuous Marks for Nominal Quantities  & Conventionally readers assume lines indicate continuous quantities and bars indicate discrete quantities. Breaking from this convention, for instance using lines for nominal measures, may cause readers to hallucinate non-existent trends based on ordering.  \cite{mcnuttlinting, zacks1999bars}\\
 \rowcolor{colorc-opaque}Modifiable Areal Unit Problem  & Spatial aggregates are often assumed as presenting their data without bias, yet they are highly dependent on the shapes of the bins defining those aggregates. This can cause readers to misunderstand the trends present in the data. \cite{fotheringham1991modifiable, kindlmann2014algebraic}\\
 \rowcolor{colorc}Charting Parameter Masking Data Error & Non-data parameters can mask a critical data errors, such as a histogram binning hiding a missing value. \cite{correll2018looks}\\
 \rowcolor{colorc-opaque}Concealed \newline Uncertainty  & Charts that don't indicate that they contain uncertainty risk giving a false impression as well a possible extreme mistrust of the data if the reader realizes the information hasn't been presented clearly. There is also a tendency to incorrectly assume that data is high quality or complete, even without evidence of this veracity. \cite{song2018s, few2019loom, mayrTrust2019, sacha2015role}\\
 \rowcolor{colorc}Manipulation of Scales & The axes and scales of a chart are presumed to straightforwardly represent quantitative information. However, manipulation of these scales (for instance, by flipping them from their commonly assumed directions, truncating or expanding them with respect to the range of the data~\cite{pandey2015deceptive, correll2017black, cleveland1982variables, ritchie2019lie, correll2019truncating}, using non-linear transforms, or employing dual axes~\cite{KindlmannAlgebraicVisPedagogyPDV2016, cairo2015graphics}) can cause viewers to misinterpret the data in a chart, for instance by exaggerating correlation~\cite{cleveland1982variables}, exaggerating effect size~\cite{correll2019truncating,pandey2015deceptive}, or misinterpreting the direction of effects~\cite{pandey2015deceptive}. \cite{cairo2015graphics,correll2017black,correll2019truncating,cleveland1982variables,KindlmannAlgebraicVisPedagogyPDV2016,pandey2015deceptive,ritchie2019lie}\\
 \rowcolor{colorc-opaque}Trend in Dual Y-Axis Charts are Arbitrary  & Multiple line series appearing on a common axis are often read as being related through an objective scaling. Yet, when y-axes are superimposed the relative selection of scaling is arbitrary, which can cause readers to misunderstand the magnitudes of relative trends. \cite{KindlmannAlgebraicVisPedagogyPDV2016, cairo2015graphics}\\
 \rowcolor{colorc}Uncorrelated Data Decorated with Best Fit Line  & We expect lines of best-fit to accurately represent data trends. Yet, if uncorrelated data is presented alongside a mark conventionally understood to indicate a trend, then a reader may falsely understand those variables as being correlated.   \\
 \rowcolor{colorc-opaque}Nominal Choropleth Conflates Color Area with Classed Statistic & Choropleth maps color spatial regions according to a value of interest. However, the size of these spatial regions may not correspond well with the actual trend in the data. For instance, U.S. Presidential election maps colored by county can communicate an incorrect impression of which candidate won the popular vote, as many counties with large area have small populations, and vice versa. \cite{gastner2005maps} CHOROPLETH CITATION?\\
 \rowcolor{colorc}Overwhelming Visual Complexity & We assume that there is a benefit to presenting all of the data in all of its complexity. However, visualizations with too much visual complexity can overwhelm or confuse the viewer and hide important trends, as with graph visualization "hairballs." \cite{hofmann2012graphical, greadability}\\

  \\\hbox{\normalsize{\textbf{READING ERRORS}}}&\\ \\
  \normalsize{Error} & \normalsize{Mirage}\\ \hline
   \rowcolor{colord}Reification (fallacy) / mistaking the map for the territory  & It can be easier to interpret a chart or map as being the information that it represents, rather than to understand that it as abstraction at the end of a causal chain. This misunderstanding can lead to falsely placed confidence in measures containing flaws or uncertainty.  CITATION PLZ\\
 \rowcolor{colord-opaque}Assumptions of Causality & We assume that highly correlated data plotted in the same graph have some important linkage. However, through visual design or arbitrary juxtaposition, viewers can come away with erroneous impressions of relation or causation of unrelated or non-causally linked variables. \cite{xiong2019illusion, few2019loom}\\
 \rowcolor{colord}Base Rate Bias & Readers assumes unexpected values in a visualization are emblematic of reliable differences. However, readers may be unaware of relevant base rates: either the relatively likelihood of what is seen as a surprising value, or, in a general sense, the false discovery rate of the entire analytic process.  \cite{correll2016surprise,pu2018garden, zgraggen2018investigating}\\
 \rowcolor{colord-opaque}Inaccessible Charts & As charts makers we often assume that our readers are homogeneous groups. Yet, the way that people read charts is heterogeneous and dependent on underlying perceptual abilities and cognitive backgrounds that can be overlooked by the designer. For instance, a viewer with color vision deficiency may interpret two colors as identical when the designer intended them to be separate. \cite{lundgard2019Sociotechnical, plaisant2005information}\\
 \rowcolor{colord}Confirmation Bias & Visualizations of data often lend an objective tone to analysis, however this medium does not prevent readers from projecting their own beliefs onto charts. This may cause readers to see trends that don't exist. \cite{valdez2017framework, few2019loom}\\
 \rowcolor{colord-opaque}Familiarity Errors: availability heuristic & Falsely extrapolate meaning from data based on completionist set of data. \cite{few2019loom}\\
 \rowcolor{colord}Anchoring Effect  & Initial framings of information tend to guide subsequent judgements. This can cause readers to place undue rhetorical weight on early observations, which may cause them to undervalue or distrust later observations.  \cite{ritchie2019lie, hullman2011visualization}\\
 \rowcolor{colord-opaque}Semmelweis Effect & We assume that readers of charts will dispassionately consume the information they contain. However, new evidence that contradicts tightly held beliefs is often discounted or discarded. \cite{valdez2017framework}\\
 \rowcolor{colord}Biases in \newline Interpretation & Each viewer arrives to a visualization with their own preconceptions, biases, and epistemic frameworks. If these biases are not carefully considered, various cognitive biases such as the backfire effect, confirmation bias, or the availability bias can cause viewers to anchor on only the data (or the reading of the data) that supports their preconceived notions, rather than a more holistic picture of the strength of the evidence. \cite{dignazio2019draft, d2016feminist,valdez2017framework, few2019loom,wall2017warning}\\
\end{longtable}
\label{table:mirage-table}
