
\begin{table*}[]
\centering
\caption{Examples of errors arising at each of the stages in our taxonomy along with the ways that those errors can manifest themselves as mirages. This list does not try to be comprehensive, only evocative.}
\small
\begin{tabular}{p{5cm}p{12cm}}
\normalsize{Error} & \normalsize{Mirage}\\ \hline
   \rowcolor{colora}\multirow{5}{0em}{\hspace{-0.6cm}\rotatebox{90}{\normalsize{Curating}}}Forgotten Population/Missing Dataset  & We expect that datasets fully cover or describe phenomena of interest. However, structural, political, and societal biases can result in over- or under-sampling of populations or problems of importance. This mismatch in coverage can hide crucial concerns about the possible scope of our analyses. \cite{missingdatasets, dignazio2019draft}\\
 \rowcolor{colora-opaque}Missing/Repeated Records  & We often assume that we have one and only one entry for each datum. However, errors in data entry or integration can result in missing or repeated values that may result in inaccurate aggregates or groupings (see \figref{fig:misspelling}). \cite{kim2003taxonomy} \\
 \rowcolor{colora}Outliers  & Many forms of analysis assume data have similar magnitudes and were generated by similar processes. Outliers, whether in the form of erroneous or unexpectedly extreme values, can greatly impact aggregation and discredit the assumptions behind many statistical tests and summaries. \cite{kim2003taxonomy} \\
 \rowcolor{colora-opaque}Spelling Mistakes  & Columns of strings are often interpreted as categorical data for the purposes of aggregation. If interpreted in this way, typos or inconsistent spelling and capitalization can create spurious categories, or remove important data from aggregate queries. \cite{wang2019uni}\\
 \rowcolor{colora}Drill-down Bias  & We assume that the order in which we investigate our data should not impact our conclusions. However, by filtering on less explanatory or relevant variables first, the full scope of the impact of later variables can be hidden. This results in insights that address only small parts of the data, when they might be true of the larger whole. \cite{lee2019avoiding}\\

   \rowcolor{colorb}\multirow{8}{0em}{\hspace{-0.6cm}\rotatebox{90}{\normalsize{Wrangling}}}Differing Number of Records by Group  & Certain summary statistics, including aggregates, are sensitive to sample size. However, the number of records aggregated into a single mark can very dramatically. This mismatch can mask this sensitivity and problematize per-mark comparisons; when combined with differing levels of aggregation, it can result in counter-intuitive results such as Simpson's Paradox. \cite{guo2017you}\\
 \rowcolor{colorb-opaque}Cherry Picking & Filtering down to an uncharacteristic subset offers an incorrect distribution. \cite{few2019loom}\\
 \rowcolor{colorb}Higher Noise than Effect Size & A high variability or noise in can mask the real information being presented, causing readers to identify non-existent trends. WICKAM LINEUPS?\\
 \rowcolor{colorb-opaque}P-Hacking & Spurious difference (due to p-hacking, say. or just un-communicated sampling variability) can give the impression of non-extant correlation. \cite{pu2018garden}\\
 \rowcolor{colorb}Outliers Combined with Wrong Aggregation Type & Using extremal aggregates (such as min/max) will likely cause misinterpretation of bars as readers tend to assume bars show sums. (I THINK THIS WHAT WITHIN BAR SAYS) \cite{newman2012bar}\\
 \rowcolor{colorb-opaque}Confusing Imputation & Imputation that generates zeroes rather removing rows can radically alter aggregates. \cite{song2018s}\\
 \rowcolor{colorb}Sampling Rate Errors  & An apparent trend may be an artifact of the sampling rate rather than the data. \cite{kindlmann2014algebraic}\\
 \rowcolor{colorb-opaque}Data Classes Distinctions Ignored & Multiple distinct data classes presented as the same can cause errors in the inference of trends. \cite{anand2015automatic}\\

   \rowcolor{colorc}\multirow{28}{0em}{\hspace{-0.6cm}\rotatebox{90}{\normalsize{Visualizing}}}Non-sequitur Visualizations  & Readers expect graphics that appear to be charts to be a mapping between data and image. Visualizations being used as decoration (in which the marks are not related to data) present non-information that might be mistaken for real information. \cite{correll2017black}\\
 \rowcolor{colorc-opaque}Banking to 45 Failure & Not using an appropriate aspect ratio can cause trends to be hallucinated in otherwise trend-free data. \cite{heer2006multi}\\
 \rowcolor{colorc}Misunderstand Area as Quantity  & The use of area encoded marks assumes readers will be able to visually compare those areas. Area encoded marks are often misunderstood as encoding length which can cause ambiguity about interpretation of magnitude. \cite{pandey2015deceptive, correll2017black}\\
 \rowcolor{colorc-opaque}Non-discriminable Colors  & The use of color as a data-encoding channel presumes the perceptual discriminability of colors. Poorly chosen color palettes, especially when marks are small or cluttered (cite), can result in ambiguity about which marks belong to which color classes. JND CITATION?\\
 \rowcolor{colorc}Overplotting  & Apparently singular marks usually represent a single value or aggregate. Yet overlapping opaque marks are read as a single mark which can cause severe misterpretations of the distribution or impose an incorrect aggregation, as in \figref{fig:opacity-permute}. CITATION PLZ\\
 \rowcolor{colorc-opaque}Singularities  & Some mark types, such as line series, are vulnerable to all of their data converging to a single point in visual space (as might occur in a parallel coordinates chart). Without graphical intervention readers can face an ambiguity in discerning paths in visual space. \cite{kindlmann2014algebraic}\\
 \rowcolor{colorc}Latent Variables Missing & ??????? never been sure what this means?  \\
 \rowcolor{colorc-opaque}Wrong/Missing Aggregation & AM: I THINK THIS IS A THING BUT WE DONT REALLY HAVE A DESCRIPTION OF IT \\
 \rowcolor{colorc}Flipped Axes  & There are conventional expections for the meaning of axis directions (e.g., time as left to right). When a design ignores those expectations, it can result in a perceived inversion of the trend.  \cite{pandey2015deceptive, correll2017black}\\
 \rowcolor{colorc-opaque}Scale Extents Larger than Range of the Data & Differences compressed and visual variability is removed. \cite{cleveland1982variables}\\
 \rowcolor{colorc}Non-linear Scales & Can cause readers to inaccurately correlate variables and cluster values. \cite{pandey2015deceptive}\\
 \rowcolor{colorc-opaque}Truncated/Expanded Axes & Axes constructed in an unintuitive manner may hide variance or cause errors in sorting marks or characterizing the distribution.  The choice of where to place the visual zero line of an axis presumes that XXX. Selections of overly truncated or expanded axes can cause exagerated effect sizes, which can lead to erroreous judgements (such as in XXX).  \cite{pandey2015deceptive, correll2017black, cleveland1982variables, ritchie2019lie, correll2019truncating}\\
 \rowcolor{colorc}Base Rate Masquerading as Data  & Visualizations comparing rates are often assumed show the relative rate, rather than the absolute rate. Yet, many displays give prominence to these absolute or base rates (such as population in choropleth maps) rather than encoded variable, causing the reader to understand this base rate as the data rate.  \cite{correll2016surprise}\\
 \rowcolor{colorc-opaque}Inappropriate Semantic Color Scale  & Data-encoded colors can collide with semantically meaningful colors, such as green on a map indicating a forest. This collision can cause readers to mistake the rendered data for associated with that color-connection. CITATION?\\
 \rowcolor{colorc}Within-bar-bias & Bar charts that have variability are frequently misunderstood. \cite{newman2012bar}\\
 \rowcolor{colorc-opaque}Highlight/Downplaying Outliers & Over highlighting outliers can mask important data features, however so can ignoring them. CITATION?\\
 \rowcolor{colorc}Clipped Outliers  & Charts are often assumed to show the full extent of their input data. A chosen domain might exclude meaningful outliers, causing some trends in the data to be invisible to the reader. \\
 \rowcolor{colorc-opaque}Continuous Marks for Nominal Quantities  & Convetionally readers assume lines indicate continuous quantities and bars indicate discrete quanties. Breaking from this convention, for instance using lines for nominal measures, may cause readers to hallucinate non-existant trends based on ordering.  \cite{mcnuttlinting, zacks1999bars}\\
 \rowcolor{colorc}Using Ordinal Measures as (Ratio/Interval) Measures & Related to "area/length mismatches" a mark might be encoded as big/medium/small which readers might then read as quantitative. \cite{stevens1946theory, few2019loom}\\
 \rowcolor{colorc-opaque}Modifiable Areal Unit Problem  & Spatial aggregates are often assumed as presenting their data without bias, yet they are highly dependent on the shapes of the bins defining those aggregates. This can cause readers to misunderstand the trends present in the data. \cite{fotheringham1991modifiable, kindlmann2014algebraic}\\
 \rowcolor{colorc}Charting Parameter Masking Data Error & Non-data parameter can mask a critical data error, such as a histogram binning hiding a missing value. \cite{correll2018looks}\\
 \rowcolor{colorc-opaque}Concealed Uncertainty & Charts that don't indicate that they contain uncertainty risk giving a false impression as well a possible extreme mistrust of the data if the reader realizes the information hasn't been presented clearly.  \cite{song2018s, few2019loom, mayrTrust2019, sacha2015role}\\
 \rowcolor{colorc}Sole Reliance on Measure of Central Tendency & Second order statistics often carry critical information about the variance or distribution, which is masked through simple central tendencies.  \cite{wall2017warning, few2019loom, matejka2017same, anscombe1973graphs}\\
 \rowcolor{colorc-opaque}Trend in Dual Y-Axis Charts are Arbitrary  & Multiple line series appearing on a common axis are often read as being related through an objective scaling. Yet, when y-axes are superimposed the relative selection of scaling is arbitrary, which can cause readers to misunderstand the magnitudes of relative trends. \cite{KindlmannAlgebraicVisPedagogyPDV2016, cairo2015graphics}\\
 \rowcolor{colorc}Uncorrelated Data Decorated with Best Fit Line  & We expect lines of best-fit to accurately represent data trends. Yet, if uncorrelated data is presented alongside a mark conventionally understood to indicate a trend, then a reader may falsely understand those variables as being correlated.   \\
 \rowcolor{colorc-opaque}Staircasing & A trend on a noisy channel may appear to reverse because of the domain selection. MAYBE A SUBSET OF CHERRY PICKING? IS THIS A KNOWN THING\\
 \rowcolor{colorc}Nominal Choropleth Conflates Color Area with Classed Statistic & Conflating class coloring and area can give a misinterpretation of base rate, as is often this case in American presidential election maps.  \cite{gastner2005maps} CHOROPLETH CITATION?\\
 \rowcolor{colorc-opaque}Presentation Masks Information & The choice of graphical rendering may obscure data found in similar encodings. For instance, some graph layouts present their data clearly while others more closely resemble hairballs. \cite{hofmann2012graphical} GRAPH CITATION\\

   \rowcolor{colord}\multirow{7}{0em}{\hspace{-0.6cm}\rotatebox{90}{\normalsize{Reading}}}Incorrectly Assumed High Quality Data/ Completeness bias & Trusting untrustworthy data cause readers to make incorrect conclusions about the information presented in a visualization. \cite{mayrTrust2019, sacha2015role}\\
 \rowcolor{colord-opaque}Unwarranted causal implications & Correlated data misunderstood (based on visual encoding) as being causally related. \cite{xiong2019illusion, few2019loom}\\
 \rowcolor{colord}Multiple Comparisons Problem & Too many iterative comparisons has high probability of generating a configuration that offers a false comparison. \cite{pu2018garden, zgraggen2018investigating}\\
 \rowcolor{colord-opaque}Not Accounting for Bias & Not addressing bias in the visual analytics process can lead analysts towards false conclusions. \cite{wall2017warning}\\
 \rowcolor{colord}Assuming View-from-Nowhere & The reader might inaccurately trust a visualization that doesn't present its origins. \cite{dignazio2019draft, d2016feminist}\\
 \rowcolor{colord-opaque}Confirmation Bias & Reader may see trend where one doesn't exist. \cite{valdez2017framework, few2019loom}\\
 \rowcolor{colord}Anchoring Effect  & Initial framings of information tend to guide subsequent judgements. This can cause readers to place undue rhetorical weight on early observations, which may cause them to undervalue or distrust later observations.  \cite{ritchie2019lie, hullman2011visualization}\\
\end{tabular}
\label{table:mirage-table}
\end{table*}
