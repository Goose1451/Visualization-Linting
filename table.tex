
\begin{table*}[]
\centering
\caption{Examples of errors arising at each of the stages in our taxonomy along with the ways that those errors can manifest themselves as mirages. This list does not try to be comprehensive, only evocative.}
\small
\begin{tabular}{p{5cm}p{12cm}}
\normalsize{Error} & \normalsize{Mirage}\\ \hline
   \rowcolor{colora}\multirow{6}{0em}{\hspace{-0.6cm}\rotatebox{90}{\normalsize{Curating}}}Forgotten Population/Missing dataset & A dataset which ignores a critical population is liable to come to conclusions that might harm that population. \cite{missingdatasets, correll2019ethical}\\
 \rowcolor{colora-opaque}Missing/Repeated Records & Faulty data can cause aggregates to be inaccurate, or at least different from the user intuition. \cite{kim2003taxonomy} \\
 \rowcolor{colora}Outliers & Outliers can cause upstream failures if not addressed properly by wrecking the visual scale and warping the perception of the real distribution.  \cite{kim2003taxonomy} \\
 \rowcolor{colora-opaque}Spelling Mistakes & Differently spelled groups can cause rows to mis-aggregate, causing inaccurate comparisons. \cite{wang2019uni}\\
 \rowcolor{colora}Unidentified Functional Dependencies & Visualization might show a strong relationship between variables, when in fact that relationship is part of the data composition (eg A+B = C, plot A vs C). \cite{wang2019uni}\\
 \rowcolor{colora-opaque}Drill-down Bias & Apparent trends might be attributed to the more specific/recently specified filters rather than relatively "simpler" explanations. \cite{lee2019avoiding}\\

   \rowcolor{colorb}\multirow{10}{0em}{\hspace{-0.6cm}\rotatebox{90}{\normalsize{Wrangling}}}Geopolitical Boundaries in Question &  \\
 \rowcolor{colorb-opaque}Differing Number of Records by Group & Marks assumed to represent consistent number of values might have a variable number of entries, which can mask missing data or can lead to incorrect assumptions about aggregates. \\
 \rowcolor{colorb}Simpson's Paradox & An observed trends reverses when the aggregation level changes. \cite{guo2017you}\\
 \rowcolor{colorb-opaque}Cherry Picking & Filtering down to an uncharacteristic subset offers an incorrect distribution. \cite{few2019loom}\\
 \rowcolor{colorb}Higher Noise than Effect Size & A high variability or noise in can mask the real information being presented, causing readers to identify non-existent trends. WICKAM LINEUPS?\\
 \rowcolor{colorb-opaque}P-Hacking & Spurious difference (due to p-hacking, say. or just un-communicated sampling variability) can give the impression of non-extant correlation. \cite{pu2018garden}\\
 \rowcolor{colorb}Outliers Combined with Wrong Aggregation Type & Using extremal aggregates (such as min/max) will likely cause misinterpretation of bars as readers tend to assume bars show sums. (I THINK THIS WHAT WITHIN BAR SAYS) \cite{newman2012bar}\\
 \rowcolor{colorb-opaque}Confusing Imputation & Imputation that generates zeroes rather removing rows can radically alter aggregates. \cite{song2018s}\\
 \rowcolor{colorb}Sampling Rate Errors  & An apparent trend may be an artifact of the sampling rate rather than the data. \cite{kindlmann2014algebraic}\\
 \rowcolor{colorb-opaque}Data Classes Distinctions Ignored & Multiple distinct data classes presented as the same can cause errors in the inference of trends. \cite{anand2015automatic}\\

   \rowcolor{colorc}\multirow{30}{0em}{\hspace{-0.6cm}\rotatebox{90}{\normalsize{Visualizing}}}Non-sequitur visualization & Visualizations being used as decoration (rather than a mapping between data and image) prey on the readers assumption that the chart is being used as normal. \cite{correll2017black}\\
 \rowcolor{colorc-opaque}Banking to 45 Failure & Not using an appropriate aspect ratio can cause trends to be hallucinated in otherwise trend-free data. \cite{heer2006multi}\\
 \rowcolor{colorc}Misunderstand Area as Quantity & Area encoded marks can be misunderstood as encoding length or area, which will describe different magnitudes of meaning (though consistent ordinal meaning). \cite{pandey2015deceptive, correll2017black}\\
 \rowcolor{colorc-opaque}Color too Close & Readers may mistake one data class for another. JND CITATION?\\
 \rowcolor{colorc}Unconventional Direction of Time  & Readers expect time to progress in the same direction as their language, reversing convention can cause reversed interpretation. \cite{correll2017black}\\
 \rowcolor{colorc-opaque}Singularities & Detail can become difficult to discern when collections of lines converge (such as in a parallel coordinates chart). \cite{kindlmann2014algebraic}\\
 \rowcolor{colorc}Improper Layering / Overplotting & Overplotting can cause an non-existent trend to emerge due to the draw order. \cite{kindlmann2014algebraic}\\
 \rowcolor{colorc-opaque}Latent Variables Missing &  \\
 \rowcolor{colorc}Wrong/Missing Aggregation & Aggregate marks with no opacity overlapping each other do an implicit max, as shown in \figref{fig:opacity-permute}. \\
 \rowcolor{colorc-opaque}Flipped Axes & A visualization presented with axes flipped from the usual direction can cause the viewer to understand the opposite message of the data. \cite{pandey2015deceptive, correll2017black, cleveland1982variables}\\
 \rowcolor{colorc}Scale Extents Larger than Range of the Data & Differences compressed and visual variability is removed. \cite{cleveland1982variables}\\
 \rowcolor{colorc-opaque}Non-linear Scales & Can cause readers to inaccurately correlate variables and cluster values. \cite{pandey2015deceptive}\\
 \rowcolor{colorc}Truncated/Expanded Axes & Axes constructed in an unintuitive manner may hide variance or cause errors in sorting marks or characterizing the distribution. \cite{pandey2015deceptive, correll2017black, cleveland1982variables, ritchie2019lie, correll2019truncating}\\
 \rowcolor{colorc-opaque}Colors Binned Unevenly & Irregular or uintutive binning may negatively affect interpretation of bin meaning. CITATION?\\
 \rowcolor{colorc}Base Rate Masquerading as Data & Mistake base rate for data signal rate, such as in maps which purport to show data but in fact just show population. \cite{correll2016surprise}\\
 \rowcolor{colorc-opaque}Inappropriate Semantic Color Scale  & Reader may misinterpret color on a map as indicating the content of that region rather than a data variable, for instance green being interpreted as indicating forests and blue indicating ocean. \\
 \rowcolor{colorc}Within-bar-bias & Bar charts that have variability are frequently misunderstood. \cite{newman2012bar}\\
 \rowcolor{colorc-opaque}Highlight/Downplaying Outliers & Over highlighting outliers can mask important data features, however so can ignoring them. CITATION?\\
 \rowcolor{colorc}Clipped Outliers & Chosen domain hides outliers, impending the reader from accurate extrema detection. \\
 \rowcolor{colorc-opaque}Continuous Marks for Nominal Quantities & Reader may hallucinate a trend based on the rendered ordering. \cite{mcnuttlinting, zacks1999bars}\\
 \rowcolor{colorc}Using Ordinal Measures as (Ratio/Interval) Measures & Related to "area/length mismatches" a mark might be encoded as big/medium/small which readers might then read as quantitative. \cite{stevens1946theory, few2019loom}\\
 \rowcolor{colorc-opaque}V-Hacking & Visualization includes arbitrary choices made by the designer that causes reader to hallucinate un-meaningful inferences. The modifiable areal unit problem is an example of this problem. \cite{fotheringham1991modifiable, kindlmann2014algebraic}\\
 \rowcolor{colorc}Charting Parameter Masking Data Error & Non-data parameter can mask a critical data error, such as a histogram binning hiding a missing value. \cite{correll2018looks}\\
 \rowcolor{colorc-opaque}Concealed Uncertainty & Charts that don't indicate that they contain uncertainty risk giving a false impression as well a possible extreme mistrust of the data if the reader realizes the information hasn't been presented clearly.  \cite{song2018s, few2019loom, mayrTrust2019, sacha2015role}\\
 \rowcolor{colorc}Sole Reliance on Measure of Central Tendency & Second order statistics often carry critical information about the variance or distribution, which is masked through simple central tendencies.  \cite{wall2017warning, few2019loom, matejka2017same, anscombe1973graphs}\\
 \rowcolor{colorc-opaque}Trend in Dual Y-Axis Charts are Arbitrary & Scales are unconnected and so any almost any relative trend can be made to appear. \cite{KindlmannAlgebraicVisPedagogyPDV2016, cairo2015graphics}\\
 \rowcolor{colorc}Uncorrelated Data Decorated with Best Fit Line & Uncareful reader might not investigate if the line of best fit actually matches with anything, instead taking a potentially incorrect trend at face value. \\
 \rowcolor{colorc-opaque}Staircasing & A trend on a noisy channel may appear to reverse because of the domain selection. MAYBE A SUBSET OF CHERRY PICKING? IS THIS A KNOWN THING\\
 \rowcolor{colorc}Nominal Choropleth Conflates Color Area with Classed Statistic & Conflating class coloring and area can give a misinterpretation of base rate, as is often this case in American presidential election maps.  \cite{gastner2005maps} CHOROPLETH CITATION?\\
 \rowcolor{colorc-opaque}Presentation Masks Information & The choice of graphical rendering may obscure data found in similar encodings. For instance, some graph layouts present their data clearly while others more closely resemble hairballs. \cite{hofmann2012graphical} GRAPH CITATION\\

   \rowcolor{colord}\multirow{9}{0em}{\hspace{-0.6cm}\rotatebox{90}{\normalsize{Reading}}}Incorrectly Assumed High Quality Data/ Completeness bias & Trusting untrustworthy data cause readers to make incorrect conclusions about the information presented in a visualization. \cite{mayrTrust2019, sacha2015role}\\
 \rowcolor{colord-opaque}Unwarranted causal implications & Correlated data misunderstood (based on visual encoding) as being causally related. \cite{xiong2019illusion, few2019loom}\\
 \rowcolor{colord}Multiple Comparisons Problem & Too many iterative comparisons has high probability of generating a configuration that offers a false comparison. \cite{pu2018garden, zgraggen2018investigating}\\
 \rowcolor{colord-opaque}Not Accounting for Bias & Not addressing bias in the visual analytics process can lead analysts towards false conclusions. \cite{wall2017warning}\\
 \rowcolor{colord}Assuming View-from-Nowhere & The reader might inaccurately trust a visualization that doesn't present its origins. \cite{dignazio2019draft}\\
 \rowcolor{colord-opaque}Over-emphasizing Data-ink Minimalism & Chart might not actually get read, minimalist charts are easier to ignore full message might get lost \cite{bateman2010useful}\\
 \rowcolor{colord}Confirmation Bias & Reader may see trend where one doesn't exist. \cite{valdez2017framework, few2019loom}\\
 \rowcolor{colord-opaque}Default Effect & The default settings of visualizations can have an outsized impact on the resulting design, potentially hiding value of interest but  also potentially guiding a user toward best practices. \cite{shah2006policy,few2019loom, hullman2011visualization}\\
 \rowcolor{colord}Anchoring Effect & Reader may be blind to changes, framing whatever they see in terms of what they've seen earlier. \cite{ritchie2019lie, hullman2011visualization}\\
\end{tabular}
\label{table:mirage-table}
\end{table*}
